\documentclass[conference]{IEEEtran}
\IEEEoverridecommandlockouts

% 基础包
\usepackage{cite}
\usepackage{amsmath,amssymb,amsfonts}
\usepackage{algorithmic}
\usepackage{graphicx}
\usepackage{textcomp}
\usepackage{xcolor}
\usepackage{booktabs}
\usepackage{multirow}
\usepackage{hyperref}
\usepackage{subcaption}

\def\BibTeX{{\rm B\kern-.05em{\sc i\kern-.025em b}\kern-.08em
    T\kern-.1667em\lower.7ex\hbox{E}\kern-.125emX}}

\begin{document}

\title{Cued-Agent: A Conformer-based Lip Reading System with Dynamic Feature Learning and Semantic Alignment}

\author{
\IEEEauthorblockN{Zezhou Wang}
\IEEEauthorblockA{\textit{School of Computer and Communication} \\
\textit{Lanzhou University of Technology}\\
Lanzhou, Gansu, China \\
wang\_ze\_zhou@163.com}
}

\maketitle

\begin{abstract}
Lip reading, also known as visual speech recognition, aims to recognize speech content from visual information of lip movements. Despite significant progress in recent years, challenges remain in capturing temporal dynamics and aligning visual-linguistic representations. In this paper, we propose Cued-Agent, a novel end-to-end lip reading system based on the Conformer architecture with three key innovations: (1) a Dynamic Feature Module that explicitly models lip movement velocity and acceleration through first and second-order derivatives; (2) a Semantic Alignment mechanism using contrastive learning to enhance visual-text feature correspondence; and (3) an Exponential Moving Average (EMA) strategy for improved model generalization. We employ a multi-task learning framework combining CTC and attention-based losses with intermediate CTC supervision. Experiments on the MVLRS dataset demonstrate that our approach achieves competitive performance with a Word Error Rate (WER) of 21.2\% and Character Error Rate (CER) of 12.7\%, validating the effectiveness of our proposed methods.
\end{abstract}

\begin{IEEEkeywords}
lip reading, visual speech recognition, Conformer, contrastive learning, dynamic features
\end{IEEEkeywords}

% 引入各个章节
\section{Introduction}
\label{sec:introduction}

Lip reading, or visual speech recognition (VSR), is the task of recognizing spoken content solely from visual information of lip movements. This technology has significant applications in human-computer interaction, assistive technologies for hearing-impaired individuals, silent speech interfaces, and multi-modal speech recognition systems \cite{afouras2018deep}.

Despite remarkable advances in deep learning-based lip reading systems, several challenges persist. First, lip movements exhibit complex temporal dynamics that are difficult to capture with standard feature extraction methods. Second, the visual-linguistic alignment problem remains challenging due to the inherent ambiguity in mapping visual lip patterns to phonetic units. Third, training deep networks for lip reading often suffers from overfitting due to limited training data.

To address these challenges, we propose \textbf{Cued-Agent}, a novel end-to-end lip reading system with three key contributions:

\begin{itemize}
    \item \textbf{Dynamic Feature Module}: We introduce a module that explicitly computes velocity (first-order derivative) and acceleration (second-order derivative) of visual features, capturing the dynamic nature of lip movements.

    \item \textbf{Semantic Alignment}: We employ contrastive learning to align visual encoder outputs with text decoder embeddings, enhancing the correspondence between visual and linguistic representations.

    \item \textbf{EMA Training Strategy}: We utilize Exponential Moving Average of model weights during training, which improves generalization and stabilizes the learning process.
\end{itemize}

Our system is built upon the Conformer architecture \cite{gulati2020conformer}, which combines the strengths of Transformers for capturing global dependencies and convolutional neural networks for modeling local patterns. We employ a multi-task learning framework that jointly optimizes CTC and attention-based losses.

The remainder of this paper is organized as follows: Section \ref{sec:related} reviews related work. Section \ref{sec:method} describes our proposed methodology. Section \ref{sec:experiments} presents experimental results. Section \ref{sec:conclusion} concludes the paper.

\section{Related Work}
\label{sec:related}

\subsection{Deep Learning for Lip Reading}

Early deep learning approaches to lip reading employed Convolutional Neural Networks (CNNs) combined with Recurrent Neural Networks (RNNs) \cite{assael2016lipnet}. LipNet \cite{assael2016lipnet} was the first end-to-end sentence-level lip reading model using spatiotemporal convolutions and bidirectional GRUs. Subsequent works improved upon this by using attention mechanisms and larger datasets \cite{afouras2018deep, chung2017lip}.

\subsection{Transformer-based Approaches}

The introduction of Transformers \cite{vaswani2017attention} revolutionized sequence modeling. In lip reading, Transformer-based models have shown superior performance by capturing long-range dependencies in video sequences \cite{afouras2018deep}. The Conformer architecture \cite{gulati2020conformer}, originally proposed for speech recognition, combines self-attention with convolution modules, making it particularly suitable for modeling both global and local patterns in lip movements.

\subsection{Multi-modal Learning}

Recent works have explored audio-visual speech recognition, leveraging both modalities for improved robustness \cite{ma2021end}. AV-HuBERT \cite{shi2022learning} introduced self-supervised pre-training for audio-visual speech recognition. Our work focuses on the visual-only setting while incorporating semantic alignment between visual and textual representations.

\subsection{Contrastive Learning}

Contrastive learning has emerged as a powerful technique for learning representations \cite{chen2020simple}. CLIP \cite{radford2021learning} demonstrated the effectiveness of contrastive learning for vision-language alignment. We adapt this approach to align visual lip features with linguistic embeddings.

\section{Methodology}
\label{sec:method}

\subsection{System Overview}

Our proposed Cued-Agent system consists of four main components: (1) a visual frontend for spatial feature extraction, (2) a Conformer encoder with dynamic feature learning, (3) a Transformer decoder for sequence generation, and (4) a semantic alignment module.

% TODO: 添加架构图后取消注释
% \begin{figure}[htbp]
% \centerline{\includegraphics[width=0.48\textwidth]{figures/architecture.pdf}}
% \caption{Overall architecture of the proposed Cued-Agent system.}
% \label{fig:architecture}
% \end{figure}

\subsection{Visual Frontend}

The visual frontend employs a 3D convolutional layer followed by a ResNet-18 backbone to extract spatiotemporal features from input video frames. Given an input video sequence $\mathbf{V} \in \mathbb{R}^{T \times C \times H \times W}$, where $T$ is the number of frames, $C$ is the number of channels, and $H, W$ are the spatial dimensions, the frontend produces feature representations:

\begin{equation}
\mathbf{F} = \text{ResNet3D}(\mathbf{V}) \in \mathbb{R}^{T' \times D}
\end{equation}

where $T'$ is the downsampled temporal length and $D$ is the feature dimension.

\subsection{Dynamic Feature Module}

A key innovation of our approach is the Dynamic Feature Module, which explicitly models the temporal dynamics of lip movements. We compute velocity (first-order derivative) and acceleration (second-order derivative) of the visual features:

\begin{equation}
\mathbf{F}^{(1)}_t = \frac{\mathbf{F}_{t+1} - \mathbf{F}_{t-1}}{2}
\end{equation}

\begin{equation}
\mathbf{F}^{(2)}_t = \frac{\mathbf{F}^{(1)}_{t+1} - \mathbf{F}^{(1)}_{t-1}}{2}
\end{equation}

The original features, velocity, and acceleration are concatenated and projected back to the original dimension:

\begin{equation}
\mathbf{F}^{dyn} = \text{Conv3D}([\mathbf{F}; \mathbf{F}^{(1)}; \mathbf{F}^{(2)}])
\end{equation}

This module is initialized with identity mapping to ensure stable training in early epochs.

\subsection{Conformer Encoder}

The Conformer encoder consists of $L=12$ layers, each containing a multi-head self-attention module and a convolution module in a Macaron-style architecture:

\begin{equation}
\mathbf{x}' = \mathbf{x} + \frac{1}{2}\text{FFN}(\mathbf{x})
\end{equation}
\begin{equation}
\mathbf{x}'' = \mathbf{x}' + \text{MHSA}(\mathbf{x}')
\end{equation}
\begin{equation}
\mathbf{x}''' = \mathbf{x}'' + \text{Conv}(\mathbf{x}'')
\end{equation}
\begin{equation}
\mathbf{y} = \text{LayerNorm}(\mathbf{x}''' + \frac{1}{2}\text{FFN}(\mathbf{x}'''))
\end{equation}

where FFN denotes the feed-forward network, MHSA is multi-head self-attention with 12 heads and dimension 768, and Conv is a depth-wise separable convolution with kernel size 31.

\subsection{Semantic Alignment Module}

To enhance the correspondence between visual and linguistic representations, we introduce a semantic alignment module based on contrastive learning. Given encoder output $\mathbf{E} \in \mathbb{R}^{B \times T \times D}$ and decoder text embeddings $\mathbf{T} \in \mathbb{R}^{B \times S \times D}$, we first apply mean pooling:

\begin{equation}
\mathbf{v}_i = \frac{1}{T}\sum_{t=1}^{T} \mathbf{E}_{i,t}, \quad \mathbf{t}_i = \frac{1}{S}\sum_{s=1}^{S} \mathbf{T}_{i,s}
\end{equation}

Then project to a shared space and normalize:

\begin{equation}
\hat{\mathbf{v}}_i = \frac{W_v \mathbf{v}_i}{\|W_v \mathbf{v}_i\|_2}, \quad \hat{\mathbf{t}}_i = \frac{W_t \mathbf{t}_i}{\|W_t \mathbf{t}_i\|_2}
\end{equation}

where $W_v, W_t \in \mathbb{R}^{256 \times D}$ are learnable projection matrices.

The contrastive loss is computed using InfoNCE:

\begin{equation}
\mathcal{L}_{v2t} = -\frac{1}{B}\sum_{i=1}^{B} \log \frac{\exp(\hat{\mathbf{v}}_i \cdot \hat{\mathbf{t}}_i / \tau)}{\sum_{j=1}^{B}\exp(\hat{\mathbf{v}}_i \cdot \hat{\mathbf{t}}_j / \tau)}
\end{equation}

\begin{equation}
\mathcal{L}_{align} = \frac{1}{2}(\mathcal{L}_{v2t} + \mathcal{L}_{t2v})
\end{equation}

where $\tau = 0.07$ is the temperature parameter.

\subsection{Multi-Task Learning Framework}

We employ a multi-task learning framework combining CTC and attention-based losses:

\begin{equation}
\mathcal{L}_{main} = \alpha \mathcal{L}_{CTC} + (1-\alpha) \mathcal{L}_{att}
\end{equation}

where $\alpha = 0.1$ balances the two losses. Additionally, we apply intermediate CTC supervision at the 6th encoder layer with weight 0.5.

The total training loss is:

\begin{equation}
\mathcal{L} = \mathcal{L}_{main} + \lambda(e) \mathcal{L}_{align}
\end{equation}

where $\lambda(e)$ is a warmup weight that linearly increases from 0 to 0.1 over the first 5 epochs.

\subsection{Exponential Moving Average}

To improve generalization, we maintain an exponential moving average of model parameters:

\begin{equation}
\theta_{EMA}^{(t)} = \beta \theta_{EMA}^{(t-1)} + (1-\beta) \theta^{(t)}
\end{equation}

where $\beta = 0.999$ is the decay rate. During validation and inference, we use $\theta_{EMA}$ instead of $\theta$.

\section{Experiments}
\label{sec:experiments}

\subsection{Dataset}

We evaluate our method on the MVLRS (Multi-View Lip Reading Sentences) dataset, which contains video clips of speakers with corresponding transcriptions. The dataset statistics are shown in Table \ref{tab:dataset}.

\begin{table}[htbp]
\caption{Dataset Statistics}
\begin{center}
\begin{tabular}{lccc}
\toprule
\textbf{Split} & \textbf{Samples} & \textbf{Avg. Length} & \textbf{Vocabulary} \\
\midrule
Train & 28,000 & 6.2s & 44 tokens \\
Validation & 1,200 & 5.8s & 44 tokens \\
Test & 1,600 & 5.9s & 44 tokens \\
\bottomrule
\end{tabular}
\label{tab:dataset}
\end{center}
\end{table}

\subsection{Implementation Details}

Our model is implemented in PyTorch with PyTorch Lightning. Key hyperparameters are listed in Table \ref{tab:hyperparams}.

\begin{table}[htbp]
\caption{Training Hyperparameters}
\begin{center}
\begin{tabular}{lc}
\toprule
\textbf{Parameter} & \textbf{Value} \\
\midrule
Encoder layers & 12 \\
Decoder layers & 6 \\
Attention dimension & 768 \\
Attention heads & 12 \\
FFN dimension & 3072 \\
Convolution kernel & 31 \\
Batch size & 2 \\
Max epochs & 75 \\
Learning rate & 1e-3 \\
Warmup epochs & 5 \\
Weight decay & 0.03 \\
EMA decay & 0.999 \\
\bottomrule
\end{tabular}
\label{tab:hyperparams}
\end{center}
\end{table}

We use AdamW optimizer with $\beta_1=0.9$, $\beta_2=0.98$. A warmup cosine learning rate schedule is employed with differential learning rates: encoder uses 0.2$\times$ base rate while decoder uses 1.0$\times$.

\subsection{Evaluation Metrics}

We use the following metrics:
\begin{itemize}
    \item \textbf{Word Error Rate (WER)}: Edit distance at word level divided by total words.
    \item \textbf{Character Error Rate (CER)}: Edit distance at character level divided by total characters.
    \item \textbf{Accuracy}: Percentage of exactly matched predictions.
\end{itemize}

\subsection{Main Results}

Table \ref{tab:results} presents the main results on the MVLRS test set.

\begin{table}[htbp]
\caption{Main Results on MVLRS Test Set}
\begin{center}
\begin{tabular}{lccc}
\toprule
\textbf{Method} & \textbf{WER(\%)} & \textbf{CER(\%)} & \textbf{Acc(\%)} \\
\midrule
LipNet \cite{assael2016lipnet} & 35.2 & 22.1 & 58.3 \\
Transformer & 28.5 & 17.8 & 68.2 \\
Conformer (baseline) & 25.1 & 15.3 & 72.5 \\
\midrule
\textbf{Cued-Agent (Ours)} & \textbf{21.2} & \textbf{12.7} & \textbf{75.0} \\
\bottomrule
\end{tabular}
\label{tab:results}
\end{center}
\end{table}

Our method achieves significant improvements over the baseline Conformer, with 3.9\% absolute reduction in WER and 2.6\% in CER.

\subsection{Ablation Study}

We conduct ablation studies to analyze the contribution of each component. Results are shown in Table \ref{tab:ablation}.

\begin{table}[htbp]
\caption{Ablation Study Results}
\begin{center}
\begin{tabular}{lccc}
\toprule
\textbf{Configuration} & \textbf{WER(\%)} & \textbf{CER(\%)} \\
\midrule
Full model & \textbf{21.2} & \textbf{12.7} \\
w/o Dynamic Features & 23.1 & 14.2 \\
w/o Semantic Alignment & 22.8 & 13.5 \\
w/o EMA & 22.4 & 13.1 \\
w/o Intermediate CTC & 23.5 & 14.0 \\
\bottomrule
\end{tabular}
\label{tab:ablation}
\end{center}
\end{table}

Key observations:
\begin{itemize}
    \item Dynamic Features contribute 1.9\% WER improvement by capturing lip movement dynamics.
    \item Semantic Alignment provides 1.6\% WER gain through better visual-text correspondence.
    \item EMA improves generalization with 1.2\% WER reduction.
    \item Intermediate CTC supervision is crucial, contributing 2.3\% WER improvement.
\end{itemize}

\section{Conclusion}
\label{sec:conclusion}

In this paper, we presented Cued-Agent, a novel end-to-end lip reading system based on the Conformer architecture. Our key contributions include:

\begin{itemize}
    \item A Dynamic Feature Module that explicitly models lip movement velocity and acceleration.
    \item A Semantic Alignment mechanism using contrastive learning for visual-text correspondence.
    \item An EMA training strategy for improved generalization.
\end{itemize}

Experiments on the MVLRS dataset demonstrate that our approach achieves competitive performance with 21.2\% WER and 12.7\% CER. Ablation studies validate the effectiveness of each proposed component.

Future work includes extending to audio-visual settings and exploring self-supervised pre-training strategies.


\bibliographystyle{IEEEtran}
\bibliography{references}

\end{document}
